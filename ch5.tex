\ifx\PREAMBLE\undefined
\documentclass[fleqn]{article}
\usepackage{geometry}
\usepackage[format = hang, font = bf]{caption}
\usepackage{subcaption}
% The following is needed in order to make the code compatible
% with both latex/dvips and pdflatex. Added for using UML generated by MetaUML.
\ifx\pdftexversion\undefined
\usepackage[dvips]{graphicx}
\else
\usepackage[pdftex]{graphicx}
\DeclareGraphicsRule{*}{mps}{*}{}
\fi
\usepackage{array}
\newcolumntype{C}[1]{>{\centering\let\newline\\\arraybackslash}b{#1}}
\newcommand{\parcell}[2][l]{\begin{tabular}{@{}#1@{}}#2\end{tabular}}
\usepackage{pdflscape}
\usepackage{multirow}
\usepackage{graphicx}
\usepackage{floatrow}
\floatsetup[table]{capposition=top}
\usepackage{bigstrut}
\usepackage{supertabular}
\usepackage{booktabs}
\usepackage{amsmath}
\usepackage{listings}
\usepackage{multicol}
\usepackage{xcolor}
\usepackage{mathrsfs}%mathcal
\usepackage{amsfonts}%allowing \mathbb{R}
\usepackage{amssymb}
\usepackage{alltt}
\usepackage{xspace}
\usepackage{color}
\usepackage{wrapfig}%text around figure
\usepackage{lipsum}
\usepackage{tikz}
\usetikzlibrary{shapes,positioning}
\usepackage{url}
\def\UrlBreaks{\do\A\do\B\do\C\do\D\do\E\do\F\do\G\do\H\do\I\do\J\do\K\do\L\do\M\do\N\do\O\do\P\do\Q\do\R\do\S\do\T\do\U\do\V\do\W\do\X\do\Y\do\Z\do\[\do\\\do\]\do\^\do\_\do\`\do\a\do\b\do\c\do\d\do\e\do\f\do\g\do\h\do\i\do\j\do\k\do\l\do\m\do\n\do\o\do\p\do\q\do\r\do\s\do\t\do\u\do\v\do\w\do\x\do\y\do\z\do\0\do\1\do\2\do\3\do\4\do\5\do\6\do\7\do\8\do\9\do\.\do\@\do\\\do\/\do\!\do\_\do\|\do\;\do\>\do\]\do\)\do\,\do\?\do\'\do+\do\=\do\#\do\-}
\usepackage{xr}%allow cross-file references
\usepackage[breaklinks = true]{hyperref}
\lstset{
language = C, 
showspaces = false,
breaklines = true, 
tabsize = 2, 
extendedchars = false, 
basicstyle = {\ttfamily \footnotesize}, 
showstringspaces=false,
keywordstyle=\color{blue!70}, 
commentstyle=\color{gray},
rulesepcolor=\color{red!20!green!20!blue!20}, 
numberstyle=\color[RGB]{0,192,192},
stringstyle=\color{red},
escapeinside={ )},
moredelim=[is][\underbar]{(**}{**)},
mathescape
}
\mathchardef\myhyphen="2D
\begin{document}
\fi
\newpage
\section{Performance Optimization}
\subsection{Performance of arithmetic operations on the reference machine}
\begin{table}[ht]
\caption{Performance \& bound of arithmetic operations}
\centering
\begin{tabular}{c|ccc|ccc}\toprule
\multirow{2}[4]*{Operation} & \multicolumn{3}{c|}{Integer} & \multicolumn{3}{c}{Floating Number}\\\cmidrule{2-7}
& Latency & Issue & Capacity & Latency & Issue & Capacity\\\midrule
Addition & 1 & 1 & 4 & 3 & 1 & 1\\
Multiplication & 3 & 1 & 1 & 5 & 1 & 2\\
Division & 3-30 & 3-30 & 1 & 3-15 & 3-15 & 1\\\midrule
\multirow{2}[4]*{Bound} & \multicolumn{3}{c|}{Integer} & \multicolumn{3}{c}{Floating Number}\\\cmidrule{2-7}
& + & * && + & *\\\midrule
Latency & 1.00 & 3.00 && 3.00 & 5.00\\
Throughput & 0.50 & 1.00 && 1.00 & 0.50\\\bottomrule
\end{tabular}
\end{table}
\subsection{Techniques}
Various techniques for performance optimization are illustrated on the following code.
\begin{lstlisting}
#define IDENT 1 //#define IDENT 0
#define OP *		//#define OP +
typedef struct {
	long len;
	data_t *data;
} vec_rec, *vec_ptr;

int get_vec_element(vec_ptr v, long index, data_t *dest) {
	if(index < 0 || index > v->len) return 0;
	*dest = v->data[index];
	return 1;
}

long vec_length(vec_ptr v) { return v->len; }

void combine1(vec_ptr v, data_t *dest) {
	long i;
	*dest = IDENT;
	for(i = 0; i < vec_length(v); ++i) {
		data_t val;
		get_vec_element(v, i, &val);
		*dest = *dest OP val;
	}
}
\end{lstlisting}
\subsubsection{Code motion}
\begin{lstlisting}
void combine2(vec_ptr v, data_t *dest) {
	long i;
	(*@\color{red}long length = vec\_length(v);@*)
	*dest = IDENT;
	for(i = 0; i < (*@\color{red}length@*); ++i) {
		data_t val;
		get_vec_element(v, i, &val);
		*dest = *dest OP val;
	}
}
\end{lstlisting}
\subsubsection{Reduce procedure calls}
\begin{lstlisting}
(*@\color{red}data\_t *get\_vec\_start(vec\_ptr v)@*) { return v->data; }

void combine3(vec_ptr v, data_t *dest) {
	long i;
	long length = vec_length(v);
	(*@\color{red}data\_t *data = get\_vec\_start(v);@*)
	*dest = IDENT;
	for(i = 0; i < length; ++i) {
		*dest = *dest OP (*@\color{red}data[i]@*);
	}
}
\end{lstlisting}
\subsubsection{Eliminate unnecessary memory references}
\begin{lstlisting}
void combine4(vec_ptr v, data_t *dest) {
	long i;
	long length = vec_length(v);
	data_t *data = get_vec_start(v);
	(*@\color{red}data\_t acc = IDENT; @*)
	for(i = 0; i < length; ++i) {
		(*@\color{red}acc@*) = (*@\color{red}acc@*) OP data[i];
	}
	(*@\color{red}*dest = acc; @*)
}
\end{lstlisting}
\subsubsection{Loop unrolling (2 * 1 unrolling)}
\begin{lstlisting}
void combine5(vec_ptr v, data_t *dest) {
	long i;
	long length = vec_length(v);
	data_t *data = get_vec_start(v);
	data_t acc = IDENT;
	(*@\color{red}long limit = length - 1;@*)//length - k + 1 for k
	for(i = 0; i < (*@\color{red}limit@*); (*@\color{red}i+=2@*)) { //i+=k for k
		(*@\color{red}acc = (acc OP data[i]) OP data[i + 1];@*)
	}
	(*@\color{red}
	for(; i < length; ++i) acc = acc OP data[i]; 
	@*)
	*dest = acc;
}
\end{lstlisting}
\subsubsection{Multiple accumulator (2 * 2 unrolling)}
\begin{lstlisting}
void combine6(vec_ptr v, data_t *dest) {
	long i;
	long length = vec_length(v);
	data_t *data = get_vec_start(v);
	long limit = length - 1;//length - k + 1 for k
	(*@\color{red}data\_t acc0 = IDENT;@*)
	(*@\color{red}data\_t acc1 = IDENT;@*)
	for(i = 0; i < limit; i+=2) { //i+=k for k
		(*@\color{red}acc0 = acc0 OP data[i];@*)
		(*@\color{red}acc1 = acc1 OP data[i + 1];@*)
	}
	for(; i < length; ++i) (*@\color{red}acc0@*) = (*@\color{red}acc0@*) OP data[i]; 
	*dest = (*@\color{red}acc0 OP acc1@*);
}
\end{lstlisting}
\subsubsection{Reassociation (2 * 1a unrolling)}
\begin{lstlisting}
void combine5(vec_ptr v, data_t *dest) {
	long i;
	long length = vec_length(v);
	data_t *data = get_vec_start(v);
	data_t acc = IDENT;
	long limit = length - 1;//length - k + 1 for k
	for(i = 0; i < limit; i+=2) { //i+=k for k
		acc = (*@\color{red}acc OP (data[i]) OP data[i + 1]@*);
	}
	for(; i < length; ++i) acc = acc OP data[i]; 
	*dest = acc;
}
\end{lstlisting}
\subsection{Limitations}
\begin{enumerate}
\item Register spilling. When the degree of parallelism $p$ exceeds the number of registers, some temporaries will have to be stored on stack, which harms the performance of the program. 
\item Penalty of wrong branch prediction. Usually we do not have to worry too much about the penalty. But we should write code suitable for conditional moves when possible.
\begin{center}
\begin{tabular}{m{0.35\textwidth}|m{0.5\textwidth}}\toprule
\multicolumn{1}{c|}{Imperative Code} & \multicolumn{1}{c}{Functional Code}\\\midrule
\begin{verbatim}
if(a[i] > b[i]) {
  long t = a[i];
  a[i] = b[i];
  b[i] = t;
}
\end{verbatim}
&
\begin{verbatim}
long min = a[i] < b[i] ? : a[i] : b[i];
long max = a[i] < b[i] ? b[i] : a[i];
a[i] = min;
b[i] = max;
\end{verbatim}
\\\midrule
\begin{verbatim}
if(src1[i1] < src2[i2])
  dest[id++] = src1[i1++];
else
  dest[id++] = src2[i2++];
\end{verbatim} 
&
\begin{verbatim}
long v1 = src[i1], v2 = src[i2];
bool choose1 = v1 < v2;
dest[id++] = choose1 ? v1 : v2;
i1 += choose1;
i2 += (1 - choose1);
\end{verbatim}
\\\bottomrule
\end{tabular} 
\end{center}
\end{enumerate}
\subsection{Memory Performance}
Try to avoid loading value from a memory position immediately after writing to it, which increases the length of the critical path. For example, this function:
\begin{lstlisting}
void psum1(float a[], float p[], long n) {
	long i;
	p[0] = a[0];
	for(i = 1; i < n; i++) {
		p[i] = p[i - 1] + a[i];//Load from p[i-1], which was written in the last iteration. 
	}
}
\end{lstlisting}
should be written as:
\begin{lstlisting}
void psum1_better(float a[], float p[], long n) {
	long i;
	float val = a[0];
	p[0] = val;
	for(i = 1; i < n; i++) {
		val += a[i];
		p[i] = val;//val is in register, not in memory.
	}
}
\end{lstlisting}
\ifx\PREAMBLE\undefined
\end{document}
\fi